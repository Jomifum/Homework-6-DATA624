% Options for packages loaded elsewhere
\PassOptionsToPackage{unicode}{hyperref}
\PassOptionsToPackage{hyphens}{url}
%
\documentclass[
]{article}
\usepackage{amsmath,amssymb}
\usepackage{iftex}
\ifPDFTeX
  \usepackage[T1]{fontenc}
  \usepackage[utf8]{inputenc}
  \usepackage{textcomp} % provide euro and other symbols
\else % if luatex or xetex
  \usepackage{unicode-math} % this also loads fontspec
  \defaultfontfeatures{Scale=MatchLowercase}
  \defaultfontfeatures[\rmfamily]{Ligatures=TeX,Scale=1}
\fi
\usepackage{lmodern}
\ifPDFTeX\else
  % xetex/luatex font selection
\fi
% Use upquote if available, for straight quotes in verbatim environments
\IfFileExists{upquote.sty}{\usepackage{upquote}}{}
\IfFileExists{microtype.sty}{% use microtype if available
  \usepackage[]{microtype}
  \UseMicrotypeSet[protrusion]{basicmath} % disable protrusion for tt fonts
}{}
\makeatletter
\@ifundefined{KOMAClassName}{% if non-KOMA class
  \IfFileExists{parskip.sty}{%
    \usepackage{parskip}
  }{% else
    \setlength{\parindent}{0pt}
    \setlength{\parskip}{6pt plus 2pt minus 1pt}}
}{% if KOMA class
  \KOMAoptions{parskip=half}}
\makeatother
\usepackage{xcolor}
\usepackage[margin=1in]{geometry}
\usepackage{color}
\usepackage{fancyvrb}
\newcommand{\VerbBar}{|}
\newcommand{\VERB}{\Verb[commandchars=\\\{\}]}
\DefineVerbatimEnvironment{Highlighting}{Verbatim}{commandchars=\\\{\}}
% Add ',fontsize=\small' for more characters per line
\usepackage{framed}
\definecolor{shadecolor}{RGB}{248,248,248}
\newenvironment{Shaded}{\begin{snugshade}}{\end{snugshade}}
\newcommand{\AlertTok}[1]{\textcolor[rgb]{0.94,0.16,0.16}{#1}}
\newcommand{\AnnotationTok}[1]{\textcolor[rgb]{0.56,0.35,0.01}{\textbf{\textit{#1}}}}
\newcommand{\AttributeTok}[1]{\textcolor[rgb]{0.13,0.29,0.53}{#1}}
\newcommand{\BaseNTok}[1]{\textcolor[rgb]{0.00,0.00,0.81}{#1}}
\newcommand{\BuiltInTok}[1]{#1}
\newcommand{\CharTok}[1]{\textcolor[rgb]{0.31,0.60,0.02}{#1}}
\newcommand{\CommentTok}[1]{\textcolor[rgb]{0.56,0.35,0.01}{\textit{#1}}}
\newcommand{\CommentVarTok}[1]{\textcolor[rgb]{0.56,0.35,0.01}{\textbf{\textit{#1}}}}
\newcommand{\ConstantTok}[1]{\textcolor[rgb]{0.56,0.35,0.01}{#1}}
\newcommand{\ControlFlowTok}[1]{\textcolor[rgb]{0.13,0.29,0.53}{\textbf{#1}}}
\newcommand{\DataTypeTok}[1]{\textcolor[rgb]{0.13,0.29,0.53}{#1}}
\newcommand{\DecValTok}[1]{\textcolor[rgb]{0.00,0.00,0.81}{#1}}
\newcommand{\DocumentationTok}[1]{\textcolor[rgb]{0.56,0.35,0.01}{\textbf{\textit{#1}}}}
\newcommand{\ErrorTok}[1]{\textcolor[rgb]{0.64,0.00,0.00}{\textbf{#1}}}
\newcommand{\ExtensionTok}[1]{#1}
\newcommand{\FloatTok}[1]{\textcolor[rgb]{0.00,0.00,0.81}{#1}}
\newcommand{\FunctionTok}[1]{\textcolor[rgb]{0.13,0.29,0.53}{\textbf{#1}}}
\newcommand{\ImportTok}[1]{#1}
\newcommand{\InformationTok}[1]{\textcolor[rgb]{0.56,0.35,0.01}{\textbf{\textit{#1}}}}
\newcommand{\KeywordTok}[1]{\textcolor[rgb]{0.13,0.29,0.53}{\textbf{#1}}}
\newcommand{\NormalTok}[1]{#1}
\newcommand{\OperatorTok}[1]{\textcolor[rgb]{0.81,0.36,0.00}{\textbf{#1}}}
\newcommand{\OtherTok}[1]{\textcolor[rgb]{0.56,0.35,0.01}{#1}}
\newcommand{\PreprocessorTok}[1]{\textcolor[rgb]{0.56,0.35,0.01}{\textit{#1}}}
\newcommand{\RegionMarkerTok}[1]{#1}
\newcommand{\SpecialCharTok}[1]{\textcolor[rgb]{0.81,0.36,0.00}{\textbf{#1}}}
\newcommand{\SpecialStringTok}[1]{\textcolor[rgb]{0.31,0.60,0.02}{#1}}
\newcommand{\StringTok}[1]{\textcolor[rgb]{0.31,0.60,0.02}{#1}}
\newcommand{\VariableTok}[1]{\textcolor[rgb]{0.00,0.00,0.00}{#1}}
\newcommand{\VerbatimStringTok}[1]{\textcolor[rgb]{0.31,0.60,0.02}{#1}}
\newcommand{\WarningTok}[1]{\textcolor[rgb]{0.56,0.35,0.01}{\textbf{\textit{#1}}}}
\usepackage{graphicx}
\makeatletter
\def\maxwidth{\ifdim\Gin@nat@width>\linewidth\linewidth\else\Gin@nat@width\fi}
\def\maxheight{\ifdim\Gin@nat@height>\textheight\textheight\else\Gin@nat@height\fi}
\makeatother
% Scale images if necessary, so that they will not overflow the page
% margins by default, and it is still possible to overwrite the defaults
% using explicit options in \includegraphics[width, height, ...]{}
\setkeys{Gin}{width=\maxwidth,height=\maxheight,keepaspectratio}
% Set default figure placement to htbp
\makeatletter
\def\fps@figure{htbp}
\makeatother
\setlength{\emergencystretch}{3em} % prevent overfull lines
\providecommand{\tightlist}{%
  \setlength{\itemsep}{0pt}\setlength{\parskip}{0pt}}
\setcounter{secnumdepth}{-\maxdimen} % remove section numbering
\ifLuaTeX
  \usepackage{selnolig}  % disable illegal ligatures
\fi
\usepackage{bookmark}
\IfFileExists{xurl.sty}{\usepackage{xurl}}{} % add URL line breaks if available
\urlstyle{same}
\hypersetup{
  pdftitle={Homework 6 D624},
  pdfauthor={Jose Fuentes},
  hidelinks,
  pdfcreator={LaTeX via pandoc}}

\title{Homework 6 D624}
\author{Jose Fuentes}
\date{2025-03-22}

\begin{document}
\maketitle

\subsection{Homework 6}\label{homework-6}

Do the exercises 9.1, 9.2, 9.3, 9.5, 9.6, 9.7, 9.8 in Hyndman. Please
submit both the Rpubs link as well as your .rmd file.

\section{9.1)}\label{section}

Figure 9.32 shows the ACFs for 36 random numbers, 360 random numbers and
1,000 random numbers.

\includegraphics[width=13.69in]{Fig932}

\begin{enumerate}
\def\labelenumi{\alph{enumi})}
\tightlist
\item
  Explain the differences among these figures. Do they all indicate that
  the data are white noise?
\end{enumerate}

The differences between ACF plots for random numbers with sample sizes
of 36, 360, and 1,000 are The variability is greater in smaller samples,
resulting in larger spikes in autocorrelation values and less stable
estimates. As the sample size increases, the ACF values converge closer
to zero, aligning with the characteristics of white noise. Despite these
differences, all plots indicate white noise as there are no systematic
patterns across the lags, and the values fall within the critical
bounds.

\begin{enumerate}
\def\labelenumi{\alph{enumi})}
\setcounter{enumi}{1}
\tightlist
\item
  Why are the critical values at different distances from the mean of
  zero? Why are the autocorrelations different in each figure when they
  each refer to white noise?
\end{enumerate}

The critical values are wider for smaller samples due to greater
variability, while larger samples lead to narrower bounds and more
precise results. Autocorrelation estimates vary because smaller samples
produce less stable ACF values, while larger samples reduce randomness
and align more closely with the expectations of white noise. In essence,
the observed differences stem from the statistical properties of varying
sample sizes.

\section{9.2)}\label{section-1}

A classic example of a non-stationary series are stock prices. Plot the
daily closing prices for Amazon stock (contained in gafa\_stock), along
with the ACF and PACF. Explain how each plot shows that the series is
non-stationary and should be differenced.

\begin{Shaded}
\begin{Highlighting}[]
\FunctionTok{library}\NormalTok{(fpp3)}
\end{Highlighting}
\end{Shaded}

\begin{verbatim}
## Warning: package 'fpp3' was built under R version 4.4.2
\end{verbatim}

\begin{verbatim}
## Registered S3 method overwritten by 'tsibble':
##   method               from 
##   as_tibble.grouped_df dplyr
\end{verbatim}

\begin{verbatim}
## -- Attaching packages -------------------------------------------- fpp3 1.0.1 --
\end{verbatim}

\begin{verbatim}
## v tibble      3.2.1     v tsibble     1.1.6
## v dplyr       1.1.4     v tsibbledata 0.4.1
## v tidyr       1.3.1     v feasts      0.4.1
## v lubridate   1.9.4     v fable       0.4.1
## v ggplot2     3.5.1
\end{verbatim}

\begin{verbatim}
## Warning: package 'dplyr' was built under R version 4.4.2
\end{verbatim}

\begin{verbatim}
## Warning: package 'tidyr' was built under R version 4.4.2
\end{verbatim}

\begin{verbatim}
## Warning: package 'lubridate' was built under R version 4.4.3
\end{verbatim}

\begin{verbatim}
## Warning: package 'ggplot2' was built under R version 4.4.3
\end{verbatim}

\begin{verbatim}
## Warning: package 'tsibble' was built under R version 4.4.2
\end{verbatim}

\begin{verbatim}
## Warning: package 'tsibbledata' was built under R version 4.4.2
\end{verbatim}

\begin{verbatim}
## Warning: package 'feasts' was built under R version 4.4.2
\end{verbatim}

\begin{verbatim}
## Warning: package 'fabletools' was built under R version 4.4.2
\end{verbatim}

\begin{verbatim}
## Warning: package 'fable' was built under R version 4.4.2
\end{verbatim}

\begin{verbatim}
## -- Conflicts ------------------------------------------------- fpp3_conflicts --
## x lubridate::date()    masks base::date()
## x dplyr::filter()      masks stats::filter()
## x tsibble::intersect() masks base::intersect()
## x tsibble::interval()  masks lubridate::interval()
## x dplyr::lag()         masks stats::lag()
## x tsibble::setdiff()   masks base::setdiff()
## x tsibble::union()     masks base::union()
\end{verbatim}

\begin{Shaded}
\begin{Highlighting}[]
\FunctionTok{library}\NormalTok{(ggplot2)}

\CommentTok{\# Plot the daily closing prices for Amazon stock}
\FunctionTok{ggplot}\NormalTok{(gafa\_stock, }\FunctionTok{aes}\NormalTok{(}\AttributeTok{x =}\NormalTok{ Date, }\AttributeTok{y =}\NormalTok{ Close)) }\SpecialCharTok{+}
  \FunctionTok{geom\_line}\NormalTok{() }\SpecialCharTok{+}
  \FunctionTok{labs}\NormalTok{(}\AttributeTok{title =} \StringTok{"Daily Closing Prices of Amazon Stock"}\NormalTok{, }\AttributeTok{x =} \StringTok{"Date"}\NormalTok{, }\AttributeTok{y =} \StringTok{"Closing Price (USD)"}\NormalTok{) }\SpecialCharTok{+}
  \FunctionTok{theme\_minimal}\NormalTok{()}
\end{Highlighting}
\end{Shaded}

\includegraphics{Assignment-6D624_files/figure-latex/92-1.pdf}

\begin{Shaded}
\begin{Highlighting}[]
\CommentTok{\# Plot the ACF of the daily closing prices}
\NormalTok{gafa\_stock }\SpecialCharTok{\%\textgreater{}\%} 
  \FunctionTok{ACF}\NormalTok{(Close) }\SpecialCharTok{\%\textgreater{}\%}
  \FunctionTok{autoplot}\NormalTok{() }\SpecialCharTok{+}
  \FunctionTok{labs}\NormalTok{(}\AttributeTok{title =} \StringTok{"ACF of Amazon Stock Closing Prices"}\NormalTok{)}
\end{Highlighting}
\end{Shaded}

\begin{verbatim}
## Warning: Provided data has an irregular interval, results should be treated
## with caution. Computing ACF by observation.
\end{verbatim}

\includegraphics{Assignment-6D624_files/figure-latex/92-2.pdf}

\begin{Shaded}
\begin{Highlighting}[]
\CommentTok{\# Plot the PACF of the daily closing prices}
\NormalTok{gafa\_stock }\SpecialCharTok{\%\textgreater{}\%}
  \FunctionTok{PACF}\NormalTok{(Close) }\SpecialCharTok{\%\textgreater{}\%}
  \FunctionTok{autoplot}\NormalTok{() }\SpecialCharTok{+}
  \FunctionTok{labs}\NormalTok{(}\AttributeTok{title =} \StringTok{"PACF of Amazon Stock Closing Prices"}\NormalTok{)}
\end{Highlighting}
\end{Shaded}

\begin{verbatim}
## Warning: Provided data has an irregular interval, results should be treated
## with caution. Computing ACF by observation.
\end{verbatim}

\includegraphics{Assignment-6D624_files/figure-latex/92-3.pdf}

\begin{enumerate}
\def\labelenumi{\arabic{enumi}.}
\tightlist
\item
  Daily Closing Prices Plot
\end{enumerate}

This plot shows Amazon's stock prices over time. The clear upward trend
and volatility indicate non-stationarity, as the mean and variance are
not constant. The presence of a trend suggests that differencing is
required to make the data stationary.\\
2. ACF Plot

The slow decay and high autocorrelation values signify that the series
is dominated by trends or patterns rather than random fluctuations.
Differencing the series to remove the trend would stabilize its mean and
variance, making it suitable for time series modeling and forecasting.

\begin{enumerate}
\def\labelenumi{\arabic{enumi}.}
\setcounter{enumi}{2}
\tightlist
\item
  The PACF plot strongly supports the conclusion that Amazon's stock
  prices are non-stationary and need differencing. First-order
  differencing will remove the trend, making the series stationary and
  suitable for time series modeling.
\end{enumerate}

\section{9.3 )}\label{section-2}

For the following series, find an appropriate Box-Cox transformation and
order of differencing in order to obtain stationary data.

\begin{enumerate}
\def\labelenumi{\alph{enumi})}
\tightlist
\item
  Turkish GDP from global\_economy.
\item
  Accommodation takings in the state of Tasmania from
  aus\_accommodation.
\item
  Monthly sales from souvenirs.
\end{enumerate}

\begin{Shaded}
\begin{Highlighting}[]
\CommentTok{\# Load necessary libraries}
\FunctionTok{library}\NormalTok{(tidyverse)}
\end{Highlighting}
\end{Shaded}

\begin{verbatim}
## Warning: package 'tidyverse' was built under R version 4.4.3
\end{verbatim}

\begin{verbatim}
## Warning: package 'readr' was built under R version 4.4.2
\end{verbatim}

\begin{verbatim}
## Warning: package 'purrr' was built under R version 4.4.3
\end{verbatim}

\begin{verbatim}
## Warning: package 'stringr' was built under R version 4.4.2
\end{verbatim}

\begin{verbatim}
## -- Attaching core tidyverse packages ------------------------ tidyverse 2.0.0 --
## v forcats 1.0.0     v readr   2.1.5
## v purrr   1.0.4     v stringr 1.5.1
## -- Conflicts ------------------------------------------ tidyverse_conflicts() --
## x dplyr::filter()     masks stats::filter()
## x tsibble::interval() masks lubridate::interval()
## x dplyr::lag()        masks stats::lag()
## i Use the conflicted package (<http://conflicted.r-lib.org/>) to force all conflicts to become errors
\end{verbatim}

\begin{Shaded}
\begin{Highlighting}[]
\FunctionTok{library}\NormalTok{(tsibble)}
\FunctionTok{library}\NormalTok{(fpp3)}
\FunctionTok{library}\NormalTok{(forecast)}
\end{Highlighting}
\end{Shaded}

\begin{verbatim}
## Warning: package 'forecast' was built under R version 4.4.3
\end{verbatim}

\begin{verbatim}
## Registered S3 method overwritten by 'quantmod':
##   method            from
##   as.zoo.data.frame zoo
\end{verbatim}

\begin{Shaded}
\begin{Highlighting}[]
\CommentTok{\# Reusable function for processing time series}
\NormalTok{process\_series }\OtherTok{\textless{}{-}} \ControlFlowTok{function}\NormalTok{(data, value\_col, time\_col, lag) \{}
  \CommentTok{\# Calculate optimal Box{-}Cox lambda}
\NormalTok{  lambda }\OtherTok{\textless{}{-}}\NormalTok{ data }\SpecialCharTok{|\textgreater{}} 
    \FunctionTok{features}\NormalTok{(\{\{ value\_col \}\}, }\AttributeTok{features =}\NormalTok{ guerrero) }\SpecialCharTok{|\textgreater{}} 
    \FunctionTok{pull}\NormalTok{(lambda\_guerrero)}
  
  \CommentTok{\# Determine number of seasonal differences}
\NormalTok{  nsdiffs }\OtherTok{\textless{}{-}}\NormalTok{ data }\SpecialCharTok{|\textgreater{}} 
    \FunctionTok{features}\NormalTok{(}\FunctionTok{box\_cox}\NormalTok{(\{\{ value\_col \}\}, lambda), }\AttributeTok{features =}\NormalTok{ unitroot\_nsdiffs) }\SpecialCharTok{|\textgreater{}} 
    \FunctionTok{pull}\NormalTok{(nsdiffs)}
  
  \CommentTok{\# If nsdiffs is less than or equal to 0, set it to 1 to avoid errors}
\NormalTok{  nsdiffs }\OtherTok{\textless{}{-}} \FunctionTok{max}\NormalTok{(nsdiffs, }\DecValTok{1}\NormalTok{)}
  
  \CommentTok{\# Determine number of first differences}
\NormalTok{  ndiffs }\OtherTok{\textless{}{-}}\NormalTok{ data }\SpecialCharTok{|\textgreater{}} 
    \FunctionTok{features}\NormalTok{(}\FunctionTok{difference}\NormalTok{(}\FunctionTok{box\_cox}\NormalTok{(\{\{ value\_col \}\}, lambda), }\AttributeTok{lag =}\NormalTok{ lag, }\AttributeTok{differences =}\NormalTok{ nsdiffs), }
             \AttributeTok{features =}\NormalTok{ unitroot\_ndiffs) }\SpecialCharTok{|\textgreater{}} 
    \FunctionTok{pull}\NormalTok{(ndiffs)}
  
  \CommentTok{\# If ndiffs is less than or equal to 0, set it to 1 to avoid errors}
\NormalTok{  ndiffs }\OtherTok{\textless{}{-}} \FunctionTok{max}\NormalTok{(ndiffs, }\DecValTok{1}\NormalTok{)}
  
  \CommentTok{\# Print results}
  \FunctionTok{cat}\NormalTok{(}\StringTok{"The optimal Box{-}Cox lambda value is"}\NormalTok{, lambda,}
      \StringTok{"}\SpecialCharTok{\textbackslash{}n}\StringTok{The optimal number of seasonal differences is"}\NormalTok{, nsdiffs,}
      \StringTok{"}\SpecialCharTok{\textbackslash{}n}\StringTok{The optimal number of first differences is"}\NormalTok{, ndiffs, }\StringTok{"}\SpecialCharTok{\textbackslash{}n}\StringTok{"}\NormalTok{)}
  
  \CommentTok{\# Generate transformed data}
  \FunctionTok{return}\NormalTok{(data }\SpecialCharTok{|\textgreater{}} 
           \FunctionTok{transmute}\NormalTok{(}
\NormalTok{             \{\{ time\_col \}\},}
             \AttributeTok{Original =}\NormalTok{ \{\{ value\_col \}\},}
             \AttributeTok{BoxCox\_Transformed =} \FunctionTok{box\_cox}\NormalTok{(\{\{ value\_col \}\}, lambda),}
             \AttributeTok{Final\_Transformed =} \FunctionTok{difference}\NormalTok{(}\FunctionTok{box\_cox}\NormalTok{(\{\{ value\_col \}\}, lambda), }\AttributeTok{lag =}\NormalTok{ lag, }\AttributeTok{differences =}\NormalTok{ nsdiffs }\SpecialCharTok{+}\NormalTok{ ndiffs)}
\NormalTok{           ))}
\NormalTok{\}}

\CommentTok{\# Turkish GDP}
\NormalTok{turkey\_gdp }\OtherTok{\textless{}{-}}\NormalTok{ global\_economy }\SpecialCharTok{|\textgreater{}} 
  \FunctionTok{filter}\NormalTok{(Country }\SpecialCharTok{==} \StringTok{"Turkey"}\NormalTok{)}

\NormalTok{turkey\_gdp\_transformed }\OtherTok{\textless{}{-}} \FunctionTok{process\_series}\NormalTok{(turkey\_gdp, GDP, Year, }\AttributeTok{lag =} \DecValTok{4}\NormalTok{)}
\end{Highlighting}
\end{Shaded}

\begin{verbatim}
## The optimal Box-Cox lambda value is 0.1572187 
## The optimal number of seasonal differences is 1 
## The optimal number of first differences is 1
\end{verbatim}

\begin{Shaded}
\begin{Highlighting}[]
\NormalTok{turkey\_gdp\_transformed }\SpecialCharTok{|\textgreater{}} 
  \FunctionTok{pivot\_longer}\NormalTok{(}\SpecialCharTok{{-}}\NormalTok{Year, }\AttributeTok{names\_to =} \StringTok{"Transformation"}\NormalTok{, }\AttributeTok{values\_to =} \StringTok{"GDP"}\NormalTok{) }\SpecialCharTok{|\textgreater{}} 
  \FunctionTok{ggplot}\NormalTok{(}\FunctionTok{aes}\NormalTok{(}\AttributeTok{x =}\NormalTok{ Year, }\AttributeTok{y =}\NormalTok{ GDP)) }\SpecialCharTok{+}
  \FunctionTok{geom\_line}\NormalTok{() }\SpecialCharTok{+}
  \FunctionTok{facet\_grid}\NormalTok{(}\FunctionTok{vars}\NormalTok{(Transformation), }\AttributeTok{scales =} \StringTok{"free\_y"}\NormalTok{) }\SpecialCharTok{+}
  \FunctionTok{labs}\NormalTok{(}\AttributeTok{title =} \StringTok{"Turkish GDP Transformations"}\NormalTok{, }\AttributeTok{y =} \StringTok{"GDP"}\NormalTok{)}
\end{Highlighting}
\end{Shaded}

\includegraphics{Assignment-6D624_files/figure-latex/93-1.pdf}

\begin{Shaded}
\begin{Highlighting}[]
\CommentTok{\# Tasmanian Accommodation}
\NormalTok{tas\_accommodation }\OtherTok{\textless{}{-}}\NormalTok{ aus\_accommodation }\SpecialCharTok{|\textgreater{}} 
  \FunctionTok{filter}\NormalTok{(State }\SpecialCharTok{==} \StringTok{"Tasmania"}\NormalTok{)}

\NormalTok{tas\_accommodation\_transformed }\OtherTok{\textless{}{-}} \FunctionTok{process\_series}\NormalTok{(tas\_accommodation, Takings, Date, }\AttributeTok{lag =} \DecValTok{4}\NormalTok{)}
\end{Highlighting}
\end{Shaded}

\begin{verbatim}
## The optimal Box-Cox lambda value is 0.001819643 
## The optimal number of seasonal differences is 1 
## The optimal number of first differences is 1
\end{verbatim}

\begin{Shaded}
\begin{Highlighting}[]
\NormalTok{tas\_accommodation\_transformed }\SpecialCharTok{|\textgreater{}} 
  \FunctionTok{pivot\_longer}\NormalTok{(}\SpecialCharTok{{-}}\NormalTok{Date, }\AttributeTok{names\_to =} \StringTok{"Transformation"}\NormalTok{, }\AttributeTok{values\_to =} \StringTok{"Takings"}\NormalTok{) }\SpecialCharTok{|\textgreater{}} 
  \FunctionTok{ggplot}\NormalTok{(}\FunctionTok{aes}\NormalTok{(}\AttributeTok{x =}\NormalTok{ Date, }\AttributeTok{y =}\NormalTok{ Takings)) }\SpecialCharTok{+}
  \FunctionTok{geom\_line}\NormalTok{() }\SpecialCharTok{+}
  \FunctionTok{facet\_grid}\NormalTok{(}\FunctionTok{vars}\NormalTok{(Transformation), }\AttributeTok{scales =} \StringTok{"free\_y"}\NormalTok{) }\SpecialCharTok{+}
  \FunctionTok{labs}\NormalTok{(}\AttributeTok{title =} \StringTok{"Tasmanian Accommodation Transformations"}\NormalTok{, }\AttributeTok{y =} \StringTok{"Takings"}\NormalTok{)}
\end{Highlighting}
\end{Shaded}

\includegraphics{Assignment-6D624_files/figure-latex/93-2.pdf}

\begin{Shaded}
\begin{Highlighting}[]
\CommentTok{\# Souvenir Sales}
\NormalTok{souvenir\_sales\_transformed }\OtherTok{\textless{}{-}} \FunctionTok{process\_series}\NormalTok{(souvenirs, Sales, Month, }\AttributeTok{lag =} \DecValTok{12}\NormalTok{)}
\end{Highlighting}
\end{Shaded}

\begin{verbatim}
## The optimal Box-Cox lambda value is 0.002118221 
## The optimal number of seasonal differences is 1 
## The optimal number of first differences is 1
\end{verbatim}

\begin{Shaded}
\begin{Highlighting}[]
\NormalTok{souvenir\_sales\_transformed }\SpecialCharTok{|\textgreater{}} 
  \FunctionTok{pivot\_longer}\NormalTok{(}\SpecialCharTok{{-}}\NormalTok{Month, }\AttributeTok{names\_to =} \StringTok{"Transformation"}\NormalTok{, }\AttributeTok{values\_to =} \StringTok{"Sales"}\NormalTok{) }\SpecialCharTok{|\textgreater{}} 
  \FunctionTok{ggplot}\NormalTok{(}\FunctionTok{aes}\NormalTok{(}\AttributeTok{x =}\NormalTok{ Month, }\AttributeTok{y =}\NormalTok{ Sales)) }\SpecialCharTok{+}
  \FunctionTok{geom\_line}\NormalTok{() }\SpecialCharTok{+}
  \FunctionTok{facet\_grid}\NormalTok{(}\FunctionTok{vars}\NormalTok{(Transformation), }\AttributeTok{scales =} \StringTok{"free\_y"}\NormalTok{) }\SpecialCharTok{+}
  \FunctionTok{labs}\NormalTok{(}\AttributeTok{title =} \StringTok{"Souvenir Sales Transformations"}\NormalTok{, }\AttributeTok{y =} \StringTok{"Sales"}\NormalTok{)}
\end{Highlighting}
\end{Shaded}

\includegraphics{Assignment-6D624_files/figure-latex/93-3.pdf}

Each dataset -- Turkish GDP, Tasmanian Accommodation Takings, and
Souvenir Sales -- initially presents challenges with trends and/or
seasonality, rendering them unsuitable for direct modeling. The
application of the Box-Cox transformation addresses variance instability
across all series, but the persistence of trends and seasonal patterns
necessitates further intervention.

Differencing techniques are then employed to eliminate these remaining
non-stationary components. For the Turkish GDP, a straightforward
differencing removes the upward trend, creating a stable series. In the
case of Tasmanian Accommodation Takings and Souvenir Sales, seasonal
differencing is applied to account for the recurring quarterly and
annual fluctuations, respectively. The combination of Box-Cox
transformation and differencing successfully stabilizes both mean and
variance, producing datasets that exhibit stationarity.

The transformation from non-stationary to stationary data is essential
for reliable time series analysis. By removing trends and seasonal
patterns, the underlying structure of the data becomes clearer, allowing
for more accurate modeling and forecasting. This process ensures that
any insights derived from the data are not confounded by the
non-stationary characteristics, leading to more robust and dependable
conclusions.

\section{9.5 )}\label{section-3}

For your retail data (from Exercise 7 in Section 2.10), find the
appropriate order of differencing (after transformation if necessary) to
obtain stationary data.

\begin{Shaded}
\begin{Highlighting}[]
\CommentTok{\# Load necessary libraries}
\FunctionTok{library}\NormalTok{(tidyverse)}
\FunctionTok{library}\NormalTok{(tsibble)}
\FunctionTok{library}\NormalTok{(fpp3)}
\FunctionTok{library}\NormalTok{(forecast)}

\CommentTok{\# Set seed for reproducibility}
\FunctionTok{set.seed}\NormalTok{(}\DecValTok{1234567}\NormalTok{)}

\CommentTok{\# Select one random series from the aus\_retail dataset}
\NormalTok{myseries }\OtherTok{\textless{}{-}}\NormalTok{ aus\_retail }\SpecialCharTok{|\textgreater{}} 
  \FunctionTok{filter}\NormalTok{(}\StringTok{\textasciigrave{}}\AttributeTok{Series ID}\StringTok{\textasciigrave{}} \SpecialCharTok{==} \FunctionTok{sample}\NormalTok{(aus\_retail}\SpecialCharTok{$}\StringTok{\textasciigrave{}}\AttributeTok{Series ID}\StringTok{\textasciigrave{}}\NormalTok{, }\DecValTok{1}\NormalTok{))}

\CommentTok{\# Display the time series for visualization}
\NormalTok{myseries }\SpecialCharTok{|\textgreater{}} 
  \FunctionTok{gg\_tsdisplay}\NormalTok{(Turnover)}
\end{Highlighting}
\end{Shaded}

\includegraphics{Assignment-6D624_files/figure-latex/95-1.pdf}

\begin{Shaded}
\begin{Highlighting}[]
\CommentTok{\# Calculate the optimal Box{-}Cox transformation lambda value}
\NormalTok{lambda }\OtherTok{\textless{}{-}}\NormalTok{ myseries }\SpecialCharTok{|\textgreater{}} 
  \FunctionTok{features}\NormalTok{(Turnover, }\AttributeTok{features =}\NormalTok{ guerrero) }\SpecialCharTok{|\textgreater{}} 
  \FunctionTok{pull}\NormalTok{(lambda\_guerrero)}

\CommentTok{\# Determine the number of seasonal differences}
\NormalTok{nsdiffs }\OtherTok{\textless{}{-}}\NormalTok{ myseries }\SpecialCharTok{|\textgreater{}} 
  \FunctionTok{features}\NormalTok{(}\FunctionTok{box\_cox}\NormalTok{(Turnover, lambda), }\AttributeTok{features =}\NormalTok{ unitroot\_nsdiffs) }\SpecialCharTok{|\textgreater{}} 
  \FunctionTok{pull}\NormalTok{(nsdiffs)}

\CommentTok{\# Ensure nsdiffs is a positive integer}
\NormalTok{nsdiffs }\OtherTok{\textless{}{-}} \FunctionTok{max}\NormalTok{(nsdiffs, }\DecValTok{1}\NormalTok{)}

\CommentTok{\# Determine the number of first differences}
\NormalTok{ndiffs }\OtherTok{\textless{}{-}}\NormalTok{ myseries }\SpecialCharTok{|\textgreater{}} 
  \FunctionTok{features}\NormalTok{(}\FunctionTok{difference}\NormalTok{(}\FunctionTok{box\_cox}\NormalTok{(Turnover, lambda), }\AttributeTok{lag =} \DecValTok{12}\NormalTok{, }\AttributeTok{differences =}\NormalTok{ nsdiffs), }
           \AttributeTok{features =}\NormalTok{ unitroot\_ndiffs) }\SpecialCharTok{|\textgreater{}} 
  \FunctionTok{pull}\NormalTok{(ndiffs)}

\CommentTok{\# Ensure ndiffs is a positive integer}
\NormalTok{ndiffs }\OtherTok{\textless{}{-}} \FunctionTok{max}\NormalTok{(ndiffs, }\DecValTok{1}\NormalTok{)}

\CommentTok{\# Print the results}
\FunctionTok{cat}\NormalTok{(}\StringTok{"The optimal Box{-}Cox lambda value for this series is"}\NormalTok{, lambda,}
    \StringTok{"}\SpecialCharTok{\textbackslash{}n}\StringTok{The optimal number of seasonal differences for this series is"}\NormalTok{, nsdiffs,}
    \StringTok{"}\SpecialCharTok{\textbackslash{}n}\StringTok{The optimal number of first differences for this series is"}\NormalTok{, ndiffs, }\StringTok{"}\SpecialCharTok{\textbackslash{}n}\StringTok{"}\NormalTok{)}
\end{Highlighting}
\end{Shaded}

\begin{verbatim}
## The optimal Box-Cox lambda value for this series is 0.1761103 
## The optimal number of seasonal differences for this series is 1 
## The optimal number of first differences for this series is 1
\end{verbatim}

\begin{Shaded}
\begin{Highlighting}[]
\CommentTok{\# Plot the fully transformed and differenced series}
\NormalTok{myseries }\SpecialCharTok{|\textgreater{}} 
  \FunctionTok{autoplot}\NormalTok{(}
    \FunctionTok{difference}\NormalTok{(}
      \FunctionTok{difference}\NormalTok{(}\FunctionTok{box\_cox}\NormalTok{(Turnover, lambda), }\AttributeTok{lag =} \DecValTok{12}\NormalTok{, }\AttributeTok{differences =}\NormalTok{ nsdiffs)}
\NormalTok{    )}
\NormalTok{  ) }\SpecialCharTok{+}
  \FunctionTok{labs}\NormalTok{(}\AttributeTok{title =} \StringTok{"Stationary Retail Turnover Series"}\NormalTok{, }\AttributeTok{y =} \StringTok{"Transformed Turnover"}\NormalTok{)}
\end{Highlighting}
\end{Shaded}

\begin{verbatim}
## Warning: Removed 13 rows containing missing values or values outside the scale range
## (`geom_line()`).
\end{verbatim}

\includegraphics{Assignment-6D624_files/figure-latex/95-2.pdf}

Time Series of Turnover: The initial turnover data exhibits a distinct
upward trend coupled with pronounced seasonal fluctuations, rendering it
non-stationary. This is corroborated by the Autocorrelation Function
(ACF) plot, which displays high autocorrelation at shorter lags,
gradually diminishing, signaling the presence of both trend and
seasonality. Additionally, the seasonal plot reveals consistent monthly
patterns, with observable peaks and troughs that align with typical
retail cycles, further emphasizing the series' non-stationary nature.

Stationary Retail Turnover Series: Following a Box-Cox transformation
and the application of seasonal and first differencing, the retail
turnover series achieves stationarity, manifesting as fluctuations
around a stable mean close to zero. The transformative process
successfully eliminates both trends and seasonal patterns, leaving
behind only short-term, random variations. Consequently, this processed
series is now appropriately conditioned for advanced statistical
modeling, including accurate forecasting techniques, as it fulfills the
criteria for stationarity required by such models.

\section{9.6 )}\label{section-4}

Simulate and plot some data from simple ARIMA models.

\begin{enumerate}
\def\labelenumi{\alph{enumi})}
\tightlist
\item
  Use the following R code to generate data from an AR(1) model with\\
  ϕ1 = 0.6 and σ2 = 1. The process starts with y1 =0.
\end{enumerate}

y \textless- numeric(100) e \textless- rnorm(100) for(i in 2:100)
y{[}i{]} \textless- 0.6*y{[}i-1{]} + e{[}i{]} sim \textless- tsibble(idx
= seq\_len(100), y = y, index = idx)

b)Produce a time plot for the series. How does the plot change as you
change ϕ1?

\begin{enumerate}
\def\labelenumi{\alph{enumi})}
\setcounter{enumi}{2}
\item
  Write your own code to generate data from an MA(1) model with θ1= 0.6
  and σ2 =1.
\item
  Produce a time plot for the series. How does the plot change as you
  change θ1?
\item
  Generate data from an ARMA(1,1) model with ϕ1 =0.6, θ1=0.6 and σ2=1.
\item
  Generate data from an AR(2) model with ϕ1= −0.8,ϕ2 =0.3 and σ2=1.(Note
  that these parameters will give a non-stationary series.)
\end{enumerate}

g)Graph the latter two series and compare them.

\begin{Shaded}
\begin{Highlighting}[]
\CommentTok{\# AR(1) Model}
\FunctionTok{library}\NormalTok{(tsibble)}
\FunctionTok{library}\NormalTok{(ggplot2)}

\CommentTok{\# Simulate AR(1) model}
\FunctionTok{set.seed}\NormalTok{(}\DecValTok{123}\NormalTok{)}
\NormalTok{y }\OtherTok{\textless{}{-}} \FunctionTok{numeric}\NormalTok{(}\DecValTok{100}\NormalTok{)}
\NormalTok{e }\OtherTok{\textless{}{-}} \FunctionTok{rnorm}\NormalTok{(}\DecValTok{100}\NormalTok{)}
\ControlFlowTok{for}\NormalTok{ (i }\ControlFlowTok{in} \DecValTok{2}\SpecialCharTok{:}\DecValTok{100}\NormalTok{) \{}
\NormalTok{  y[i] }\OtherTok{\textless{}{-}} \FloatTok{0.6} \SpecialCharTok{*}\NormalTok{ y[i }\SpecialCharTok{{-}} \DecValTok{1}\NormalTok{] }\SpecialCharTok{+}\NormalTok{ e[i] }\CommentTok{\# AR(1) formula}
\NormalTok{\}}
\NormalTok{sim\_ar1 }\OtherTok{\textless{}{-}} \FunctionTok{tsibble}\NormalTok{(}\AttributeTok{idx =} \FunctionTok{seq\_len}\NormalTok{(}\DecValTok{100}\NormalTok{), }\AttributeTok{y =}\NormalTok{ y, }\AttributeTok{index =}\NormalTok{ idx)}

\CommentTok{\# Time plot for AR(1)}
\FunctionTok{ggplot}\NormalTok{(sim\_ar1, }\FunctionTok{aes}\NormalTok{(}\AttributeTok{x =}\NormalTok{ idx, }\AttributeTok{y =}\NormalTok{ y)) }\SpecialCharTok{+}
  \FunctionTok{geom\_line}\NormalTok{(}\AttributeTok{color =} \StringTok{"blue"}\NormalTok{) }\SpecialCharTok{+}
  \FunctionTok{labs}\NormalTok{(}\AttributeTok{title =} \StringTok{"AR(1) Model Simulation"}\NormalTok{, }\AttributeTok{x =} \StringTok{"Time"}\NormalTok{, }\AttributeTok{y =} \StringTok{"Value"}\NormalTok{) }\SpecialCharTok{+}
  \FunctionTok{theme\_minimal}\NormalTok{()}
\end{Highlighting}
\end{Shaded}

\includegraphics{Assignment-6D624_files/figure-latex/96-1.pdf}

\begin{Shaded}
\begin{Highlighting}[]
\CommentTok{\# MA(1) Model}
\CommentTok{\# Simulate MA(1) model}
\FunctionTok{set.seed}\NormalTok{(}\DecValTok{123}\NormalTok{)}
\NormalTok{e }\OtherTok{\textless{}{-}} \FunctionTok{rnorm}\NormalTok{(}\DecValTok{101}\NormalTok{) }\CommentTok{\# Generate errors}
\NormalTok{y }\OtherTok{\textless{}{-}} \FunctionTok{numeric}\NormalTok{(}\DecValTok{100}\NormalTok{)}
\ControlFlowTok{for}\NormalTok{ (i }\ControlFlowTok{in} \DecValTok{2}\SpecialCharTok{:}\DecValTok{100}\NormalTok{) \{}
\NormalTok{  y[i] }\OtherTok{\textless{}{-}}\NormalTok{ e[i] }\SpecialCharTok{+} \FloatTok{0.6} \SpecialCharTok{*}\NormalTok{ e[i }\SpecialCharTok{{-}} \DecValTok{1}\NormalTok{] }\CommentTok{\# MA(1) formula}
\NormalTok{\}}
\NormalTok{sim\_ma1 }\OtherTok{\textless{}{-}} \FunctionTok{tsibble}\NormalTok{(}\AttributeTok{idx =} \FunctionTok{seq\_len}\NormalTok{(}\DecValTok{100}\NormalTok{), }\AttributeTok{y =}\NormalTok{ y, }\AttributeTok{index =}\NormalTok{ idx)}

\CommentTok{\# Time plot for MA(1)}
\FunctionTok{ggplot}\NormalTok{(sim\_ma1, }\FunctionTok{aes}\NormalTok{(}\AttributeTok{x =}\NormalTok{ idx, }\AttributeTok{y =}\NormalTok{ y)) }\SpecialCharTok{+}
  \FunctionTok{geom\_line}\NormalTok{(}\AttributeTok{color =} \StringTok{"red"}\NormalTok{) }\SpecialCharTok{+}
  \FunctionTok{labs}\NormalTok{(}\AttributeTok{title =} \StringTok{"MA(1) Model Simulation"}\NormalTok{, }\AttributeTok{x =} \StringTok{"Time"}\NormalTok{, }\AttributeTok{y =} \StringTok{"Value"}\NormalTok{) }\SpecialCharTok{+}
  \FunctionTok{theme\_minimal}\NormalTok{()}
\end{Highlighting}
\end{Shaded}

\includegraphics{Assignment-6D624_files/figure-latex/96-2.pdf}

\begin{Shaded}
\begin{Highlighting}[]
\CommentTok{\#ARMA(1,1) Model}
\CommentTok{\# Simulate ARMA(1,1) model}
\FunctionTok{set.seed}\NormalTok{(}\DecValTok{123}\NormalTok{)}
\NormalTok{y }\OtherTok{\textless{}{-}} \FunctionTok{numeric}\NormalTok{(}\DecValTok{100}\NormalTok{)}
\NormalTok{e }\OtherTok{\textless{}{-}} \FunctionTok{rnorm}\NormalTok{(}\DecValTok{100}\NormalTok{)}
\ControlFlowTok{for}\NormalTok{ (i }\ControlFlowTok{in} \DecValTok{2}\SpecialCharTok{:}\DecValTok{100}\NormalTok{) \{}
\NormalTok{  y[i] }\OtherTok{\textless{}{-}} \FloatTok{0.6} \SpecialCharTok{*}\NormalTok{ y[i }\SpecialCharTok{{-}} \DecValTok{1}\NormalTok{] }\SpecialCharTok{+}\NormalTok{ e[i] }\SpecialCharTok{+} \FloatTok{0.6} \SpecialCharTok{*}\NormalTok{ e[i }\SpecialCharTok{{-}} \DecValTok{1}\NormalTok{] }\CommentTok{\# ARMA(1,1) formula}
\NormalTok{\}}
\NormalTok{sim\_arma11 }\OtherTok{\textless{}{-}} \FunctionTok{tsibble}\NormalTok{(}\AttributeTok{idx =} \FunctionTok{seq\_len}\NormalTok{(}\DecValTok{100}\NormalTok{), }\AttributeTok{y =}\NormalTok{ y, }\AttributeTok{index =}\NormalTok{ idx)}

\CommentTok{\# Time plot for ARMA(1,1)}
\FunctionTok{ggplot}\NormalTok{(sim\_arma11, }\FunctionTok{aes}\NormalTok{(}\AttributeTok{x =}\NormalTok{ idx, }\AttributeTok{y =}\NormalTok{ y)) }\SpecialCharTok{+}
  \FunctionTok{geom\_line}\NormalTok{(}\AttributeTok{color =} \StringTok{"green"}\NormalTok{) }\SpecialCharTok{+}
  \FunctionTok{labs}\NormalTok{(}\AttributeTok{title =} \StringTok{"ARMA(1,1) Model Simulation"}\NormalTok{, }\AttributeTok{x =} \StringTok{"Time"}\NormalTok{, }\AttributeTok{y =} \StringTok{"Value"}\NormalTok{) }\SpecialCharTok{+}
  \FunctionTok{theme\_minimal}\NormalTok{()}
\end{Highlighting}
\end{Shaded}

\includegraphics{Assignment-6D624_files/figure-latex/96-3.pdf}

\begin{Shaded}
\begin{Highlighting}[]
\CommentTok{\#AR(2) Model}

\CommentTok{\# Simulate AR(2) model}
\FunctionTok{set.seed}\NormalTok{(}\DecValTok{123}\NormalTok{)}
\NormalTok{y }\OtherTok{\textless{}{-}} \FunctionTok{numeric}\NormalTok{(}\DecValTok{100}\NormalTok{)}
\NormalTok{e }\OtherTok{\textless{}{-}} \FunctionTok{rnorm}\NormalTok{(}\DecValTok{100}\NormalTok{)}
\ControlFlowTok{for}\NormalTok{ (i }\ControlFlowTok{in} \DecValTok{3}\SpecialCharTok{:}\DecValTok{100}\NormalTok{) \{}
\NormalTok{  y[i] }\OtherTok{\textless{}{-}} \SpecialCharTok{{-}}\FloatTok{0.8} \SpecialCharTok{*}\NormalTok{ y[i }\SpecialCharTok{{-}} \DecValTok{1}\NormalTok{] }\SpecialCharTok{+} \FloatTok{0.3} \SpecialCharTok{*}\NormalTok{ y[i }\SpecialCharTok{{-}} \DecValTok{2}\NormalTok{] }\SpecialCharTok{+}\NormalTok{ e[i] }\CommentTok{\# AR(2) formula}
\NormalTok{\}}
\NormalTok{sim\_ar2 }\OtherTok{\textless{}{-}} \FunctionTok{tsibble}\NormalTok{(}\AttributeTok{idx =} \FunctionTok{seq\_len}\NormalTok{(}\DecValTok{100}\NormalTok{), }\AttributeTok{y =}\NormalTok{ y, }\AttributeTok{index =}\NormalTok{ idx)}

\CommentTok{\# Time plot for AR(2)}
\FunctionTok{ggplot}\NormalTok{(sim\_ar2, }\FunctionTok{aes}\NormalTok{(}\AttributeTok{x =}\NormalTok{ idx, }\AttributeTok{y =}\NormalTok{ y)) }\SpecialCharTok{+}
  \FunctionTok{geom\_line}\NormalTok{(}\AttributeTok{color =} \StringTok{"purple"}\NormalTok{) }\SpecialCharTok{+}
  \FunctionTok{labs}\NormalTok{(}\AttributeTok{title =} \StringTok{"AR(2) Model Simulation"}\NormalTok{, }\AttributeTok{x =} \StringTok{"Time"}\NormalTok{, }\AttributeTok{y =} \StringTok{"Value"}\NormalTok{) }\SpecialCharTok{+}
  \FunctionTok{theme\_minimal}\NormalTok{()}
\end{Highlighting}
\end{Shaded}

\includegraphics{Assignment-6D624_files/figure-latex/96-4.pdf}

\section{9.7 )}\label{section-5}

Consider aus\_airpassengers, the total number of passengers (in
millions) from Australian air carriers for the period 1970-2011.

\begin{enumerate}
\def\labelenumi{\alph{enumi}.}
\tightlist
\item
  Use ARIMA() to find an appropriate ARIMA model. What model was
  selected. Check that the residuals look like white noise. Plot
  forecasts for the next 10 periods.
\item
  Write the model in terms of the backshift operator.
\item
  Plot forecasts from an ARIMA(0,1,0) model with drift and compare these
  to part a.
\item
  Plot forecasts from an ARIMA(2,1,2) model with drift and compare these
  to parts a and c.~Remove the constant and see what happens.
\item
  Plot forecasts from an ARIMA(0,2,1) model with a constant. What
  happens?
\end{enumerate}

\begin{Shaded}
\begin{Highlighting}[]
\CommentTok{\# Load necessary libraries}
\FunctionTok{library}\NormalTok{(fpp3)}
\FunctionTok{library}\NormalTok{(ggplot2)}

\CommentTok{\# Load the aus\_airpassengers dataset}
\NormalTok{data }\OtherTok{\textless{}{-}}\NormalTok{ aus\_airpassengers}

\CommentTok{\# Part (a) Find an appropriate ARIMA model}
\NormalTok{model\_a }\OtherTok{\textless{}{-}}\NormalTok{ data }\SpecialCharTok{|\textgreater{}} \FunctionTok{model}\NormalTok{(}\FunctionTok{ARIMA}\NormalTok{(Passengers))}
\FunctionTok{report}\NormalTok{(model\_a)}
\end{Highlighting}
\end{Shaded}

\begin{verbatim}
## Series: Passengers 
## Model: ARIMA(0,2,1) 
## 
## Coefficients:
##           ma1
##       -0.8963
## s.e.   0.0594
## 
## sigma^2 estimated as 4.308:  log likelihood=-97.02
## AIC=198.04   AICc=198.32   BIC=201.65
\end{verbatim}

\begin{Shaded}
\begin{Highlighting}[]
\CommentTok{\# Check residuals}
\NormalTok{residuals\_a }\OtherTok{\textless{}{-}} \FunctionTok{augment}\NormalTok{(model\_a)}
\FunctionTok{autoplot}\NormalTok{(residuals\_a, .resid) }\SpecialCharTok{+}
  \FunctionTok{labs}\NormalTok{(}\AttributeTok{title =} \StringTok{"Residuals of Fitted ARIMA Model"}\NormalTok{, }\AttributeTok{y =} \StringTok{"Residuals"}\NormalTok{)}
\end{Highlighting}
\end{Shaded}

\includegraphics{Assignment-6D624_files/figure-latex/97-1.pdf}

\begin{Shaded}
\begin{Highlighting}[]
\FunctionTok{ggAcf}\NormalTok{(residuals\_a}\SpecialCharTok{$}\NormalTok{.resid) }\SpecialCharTok{+}
  \FunctionTok{labs}\NormalTok{(}\AttributeTok{title =} \StringTok{"ACF of Residuals"}\NormalTok{)}
\end{Highlighting}
\end{Shaded}

\includegraphics{Assignment-6D624_files/figure-latex/97-2.pdf}

\begin{Shaded}
\begin{Highlighting}[]
\CommentTok{\# Forecast for the next 10 periods}
\NormalTok{forecast\_a }\OtherTok{\textless{}{-}}\NormalTok{ model\_a }\SpecialCharTok{|\textgreater{}} \FunctionTok{forecast}\NormalTok{(}\AttributeTok{h =} \DecValTok{10}\NormalTok{)}

\CommentTok{\# Plot and save forecast (Part a)}
\NormalTok{p\_a }\OtherTok{\textless{}{-}} \FunctionTok{autoplot}\NormalTok{(forecast\_a, data) }\SpecialCharTok{+}
  \FunctionTok{labs}\NormalTok{(}\AttributeTok{title =} \StringTok{"ARIMA Model Forecast (Part A)"}\NormalTok{, }\AttributeTok{y =} \StringTok{"Passengers"}\NormalTok{)}
\FunctionTok{print}\NormalTok{(p\_a) }\CommentTok{\# Display plot in console}
\end{Highlighting}
\end{Shaded}

\includegraphics{Assignment-6D624_files/figure-latex/97-3.pdf}

\begin{Shaded}
\begin{Highlighting}[]
\FunctionTok{ggsave}\NormalTok{(}\StringTok{"forecast\_a.png"}\NormalTok{, }\AttributeTok{plot =}\NormalTok{ p\_a)}
\end{Highlighting}
\end{Shaded}

\begin{verbatim}
## Saving 6.5 x 4.5 in image
\end{verbatim}

\begin{Shaded}
\begin{Highlighting}[]
\CommentTok{\# Part (b) Write the model in backshift operator terms}
\FunctionTok{cat}\NormalTok{(}\StringTok{"The ARIMA model (selected in Part A) in terms of the backshift operator is printed by \textquotesingle{}report()\textquotesingle{} output.}\SpecialCharTok{\textbackslash{}n}\StringTok{"}\NormalTok{)}
\end{Highlighting}
\end{Shaded}

\begin{verbatim}
## The ARIMA model (selected in Part A) in terms of the backshift operator is printed by 'report()' output.
\end{verbatim}

\begin{Shaded}
\begin{Highlighting}[]
\CommentTok{\# Part (c) ARIMA(0,1,0) with drift}
\NormalTok{model\_c }\OtherTok{\textless{}{-}}\NormalTok{ data }\SpecialCharTok{|\textgreater{}} \FunctionTok{model}\NormalTok{(}\FunctionTok{ARIMA}\NormalTok{(Passengers }\SpecialCharTok{\textasciitilde{}} \FunctionTok{drift}\NormalTok{()))}
\end{Highlighting}
\end{Shaded}

\begin{verbatim}
## Warning: 1 error encountered for ARIMA(Passengers ~ drift())
## [1] could not find function "drift"
\end{verbatim}

\begin{Shaded}
\begin{Highlighting}[]
\NormalTok{forecast\_c }\OtherTok{\textless{}{-}}\NormalTok{ model\_c }\SpecialCharTok{|\textgreater{}} \FunctionTok{forecast}\NormalTok{(}\AttributeTok{h =} \DecValTok{10}\NormalTok{)}

\CommentTok{\# Plot and save forecast (Part c)}
\NormalTok{p\_c }\OtherTok{\textless{}{-}} \FunctionTok{autoplot}\NormalTok{(forecast\_c, data) }\SpecialCharTok{+}
  \FunctionTok{labs}\NormalTok{(}\AttributeTok{title =} \StringTok{"ARIMA(0,1,0) Model with Drift"}\NormalTok{, }\AttributeTok{y =} \StringTok{"Passengers"}\NormalTok{)}
\FunctionTok{print}\NormalTok{(p\_c) }\CommentTok{\# Display plot in console}
\end{Highlighting}
\end{Shaded}

\begin{verbatim}
## Warning in max(ids, na.rm = TRUE): no non-missing arguments to max; returning
## -Inf
\end{verbatim}

\begin{verbatim}
## Warning in max(ids, na.rm = TRUE): no non-missing arguments to max; returning
## -Inf
\end{verbatim}

\begin{verbatim}
## Warning: Removed 10 rows containing missing values or values outside the scale range
## (`geom_line()`).
\end{verbatim}

\includegraphics{Assignment-6D624_files/figure-latex/97-4.pdf}

\begin{Shaded}
\begin{Highlighting}[]
\FunctionTok{ggsave}\NormalTok{(}\StringTok{"forecast\_c.png"}\NormalTok{, }\AttributeTok{plot =}\NormalTok{ p\_c)}
\end{Highlighting}
\end{Shaded}

\begin{verbatim}
## Saving 6.5 x 4.5 in image
\end{verbatim}

\begin{verbatim}
## Warning in max(ids, na.rm = TRUE): no non-missing arguments to max; returning
## -Inf
\end{verbatim}

\begin{verbatim}
## Warning in max(ids, na.rm = TRUE): no non-missing arguments to max; returning
## -Inf
\end{verbatim}

\begin{verbatim}
## Warning: Removed 10 rows containing missing values or values outside the scale range
## (`geom_line()`).
\end{verbatim}

\begin{Shaded}
\begin{Highlighting}[]
\CommentTok{\# Part (d) ARIMA(2,1,2) with drift}
\NormalTok{model\_d }\OtherTok{\textless{}{-}}\NormalTok{ data }\SpecialCharTok{|\textgreater{}} \FunctionTok{model}\NormalTok{(}\FunctionTok{ARIMA}\NormalTok{(Passengers }\SpecialCharTok{\textasciitilde{}} \FunctionTok{pdq}\NormalTok{(}\DecValTok{2}\NormalTok{,}\DecValTok{1}\NormalTok{,}\DecValTok{2}\NormalTok{) }\SpecialCharTok{+} \FunctionTok{drift}\NormalTok{()))}
\end{Highlighting}
\end{Shaded}

\begin{verbatim}
## Warning: 1 error encountered for ARIMA(Passengers ~ pdq(2, 1, 2) + drift())
## [1] could not find function "drift"
\end{verbatim}

\begin{Shaded}
\begin{Highlighting}[]
\NormalTok{forecast\_d }\OtherTok{\textless{}{-}}\NormalTok{ model\_d }\SpecialCharTok{|\textgreater{}} \FunctionTok{forecast}\NormalTok{(}\AttributeTok{h =} \DecValTok{10}\NormalTok{)}

\CommentTok{\# Plot and save forecast (Part d with constant)}
\NormalTok{p\_d }\OtherTok{\textless{}{-}} \FunctionTok{autoplot}\NormalTok{(forecast\_d, data) }\SpecialCharTok{+}
  \FunctionTok{labs}\NormalTok{(}\AttributeTok{title =} \StringTok{"ARIMA(2,1,2) Model with Drift"}\NormalTok{, }\AttributeTok{y =} \StringTok{"Passengers"}\NormalTok{)}
\FunctionTok{print}\NormalTok{(p\_d) }\CommentTok{\# Display plot in console}
\end{Highlighting}
\end{Shaded}

\begin{verbatim}
## Warning in max(ids, na.rm = TRUE): no non-missing arguments to max; returning
## -Inf
\end{verbatim}

\begin{verbatim}
## Warning in max(ids, na.rm = TRUE): no non-missing arguments to max; returning
## -Inf
\end{verbatim}

\begin{verbatim}
## Warning: Removed 10 rows containing missing values or values outside the scale range
## (`geom_line()`).
\end{verbatim}

\includegraphics{Assignment-6D624_files/figure-latex/97-5.pdf}

\begin{Shaded}
\begin{Highlighting}[]
\FunctionTok{ggsave}\NormalTok{(}\StringTok{"forecast\_d\_with\_constant.png"}\NormalTok{, }\AttributeTok{plot =}\NormalTok{ p\_d)}
\end{Highlighting}
\end{Shaded}

\begin{verbatim}
## Saving 6.5 x 4.5 in image
\end{verbatim}

\begin{verbatim}
## Warning in max(ids, na.rm = TRUE): no non-missing arguments to max; returning
## -Inf
\end{verbatim}

\begin{verbatim}
## Warning in max(ids, na.rm = TRUE): no non-missing arguments to max; returning
## -Inf
\end{verbatim}

\begin{verbatim}
## Warning: Removed 10 rows containing missing values or values outside the scale range
## (`geom_line()`).
\end{verbatim}

\begin{Shaded}
\begin{Highlighting}[]
\CommentTok{\# Remove the constant (drift) and observe behavior}
\NormalTok{model\_d\_no\_constant }\OtherTok{\textless{}{-}}\NormalTok{ data }\SpecialCharTok{|\textgreater{}} \FunctionTok{model}\NormalTok{(}\FunctionTok{ARIMA}\NormalTok{(Passengers }\SpecialCharTok{\textasciitilde{}} \FunctionTok{pdq}\NormalTok{(}\DecValTok{2}\NormalTok{,}\DecValTok{1}\NormalTok{,}\DecValTok{2}\NormalTok{)))}
\end{Highlighting}
\end{Shaded}

\begin{verbatim}
## Warning: It looks like you're trying to fully specify your ARIMA model but have not said if a constant should be included.
## You can include a constant using `ARIMA(y~1)` to the formula or exclude it by adding `ARIMA(y~0)`.
\end{verbatim}

\begin{verbatim}
## Warning: 1 error encountered for ARIMA(Passengers ~ pdq(2, 1, 2))
## [1] Could not find an appropriate ARIMA model.
## This is likely because automatic selection does not select models with characteristic roots that may be numerically unstable.
## For more details, refer to https://otexts.com/fpp3/arima-r.html#plotting-the-characteristic-roots
\end{verbatim}

\begin{Shaded}
\begin{Highlighting}[]
\NormalTok{forecast\_d\_no\_constant }\OtherTok{\textless{}{-}}\NormalTok{ model\_d\_no\_constant }\SpecialCharTok{|\textgreater{}} \FunctionTok{forecast}\NormalTok{(}\AttributeTok{h =} \DecValTok{10}\NormalTok{)}

\CommentTok{\# Plot and save forecast (Part d without constant)}
\NormalTok{p\_d\_nc }\OtherTok{\textless{}{-}} \FunctionTok{autoplot}\NormalTok{(forecast\_d\_no\_constant, data) }\SpecialCharTok{+}
  \FunctionTok{labs}\NormalTok{(}\AttributeTok{title =} \StringTok{"ARIMA(2,1,2) Model Without Drift"}\NormalTok{, }\AttributeTok{y =} \StringTok{"Passengers"}\NormalTok{)}
\FunctionTok{print}\NormalTok{(p\_d\_nc) }\CommentTok{\# Display plot in console}
\end{Highlighting}
\end{Shaded}

\begin{verbatim}
## Warning in max(ids, na.rm = TRUE): no non-missing arguments to max; returning
## -Inf
\end{verbatim}

\begin{verbatim}
## Warning in max(ids, na.rm = TRUE): no non-missing arguments to max; returning
## -Inf
\end{verbatim}

\begin{verbatim}
## Warning: Removed 10 rows containing missing values or values outside the scale range
## (`geom_line()`).
\end{verbatim}

\includegraphics{Assignment-6D624_files/figure-latex/97-6.pdf}

\begin{Shaded}
\begin{Highlighting}[]
\FunctionTok{ggsave}\NormalTok{(}\StringTok{"forecast\_d\_no\_constant.png"}\NormalTok{, }\AttributeTok{plot =}\NormalTok{ p\_d\_nc)}
\end{Highlighting}
\end{Shaded}

\begin{verbatim}
## Saving 6.5 x 4.5 in image
\end{verbatim}

\begin{verbatim}
## Warning in max(ids, na.rm = TRUE): no non-missing arguments to max; returning
## -Inf
\end{verbatim}

\begin{verbatim}
## Warning in max(ids, na.rm = TRUE): no non-missing arguments to max; returning
## -Inf
\end{verbatim}

\begin{verbatim}
## Warning: Removed 10 rows containing missing values or values outside the scale range
## (`geom_line()`).
\end{verbatim}

\begin{Shaded}
\begin{Highlighting}[]
\CommentTok{\# Part (e) ARIMA(0,2,1) with a constant}
\NormalTok{model\_e }\OtherTok{\textless{}{-}}\NormalTok{ data }\SpecialCharTok{|\textgreater{}} \FunctionTok{model}\NormalTok{(}\FunctionTok{ARIMA}\NormalTok{(Passengers }\SpecialCharTok{\textasciitilde{}} \FunctionTok{pdq}\NormalTok{(}\DecValTok{0}\NormalTok{,}\DecValTok{2}\NormalTok{,}\DecValTok{1}\NormalTok{) }\SpecialCharTok{+} \FunctionTok{drift}\NormalTok{()))}
\end{Highlighting}
\end{Shaded}

\begin{verbatim}
## Warning: 1 error encountered for ARIMA(Passengers ~ pdq(0, 2, 1) + drift())
## [1] could not find function "drift"
\end{verbatim}

\begin{Shaded}
\begin{Highlighting}[]
\NormalTok{forecast\_e }\OtherTok{\textless{}{-}}\NormalTok{ model\_e }\SpecialCharTok{|\textgreater{}} \FunctionTok{forecast}\NormalTok{(}\AttributeTok{h =} \DecValTok{10}\NormalTok{)}

\CommentTok{\# Plot and save forecast (Part e)}
\NormalTok{p\_e }\OtherTok{\textless{}{-}} \FunctionTok{autoplot}\NormalTok{(forecast\_e, data) }\SpecialCharTok{+}
  \FunctionTok{labs}\NormalTok{(}\AttributeTok{title =} \StringTok{"ARIMA(0,2,1) Model with Constant"}\NormalTok{, }\AttributeTok{y =} \StringTok{"Passengers"}\NormalTok{)}
\FunctionTok{print}\NormalTok{(p\_e) }\CommentTok{\# Display plot in console}
\end{Highlighting}
\end{Shaded}

\begin{verbatim}
## Warning in max(ids, na.rm = TRUE): no non-missing arguments to max; returning
## -Inf
\end{verbatim}

\begin{verbatim}
## Warning in max(ids, na.rm = TRUE): no non-missing arguments to max; returning
## -Inf
\end{verbatim}

\begin{verbatim}
## Warning: Removed 10 rows containing missing values or values outside the scale range
## (`geom_line()`).
\end{verbatim}

\includegraphics{Assignment-6D624_files/figure-latex/97-7.pdf}

\begin{Shaded}
\begin{Highlighting}[]
\FunctionTok{ggsave}\NormalTok{(}\StringTok{"forecast\_e.png"}\NormalTok{, }\AttributeTok{plot =}\NormalTok{ p\_e)}
\end{Highlighting}
\end{Shaded}

\begin{verbatim}
## Saving 6.5 x 4.5 in image
\end{verbatim}

\begin{verbatim}
## Warning in max(ids, na.rm = TRUE): no non-missing arguments to max; returning
## -Inf
\end{verbatim}

\begin{verbatim}
## Warning in max(ids, na.rm = TRUE): no non-missing arguments to max; returning
## -Inf
\end{verbatim}

\begin{verbatim}
## Warning: Removed 10 rows containing missing values or values outside the scale range
## (`geom_line()`).
\end{verbatim}

\begin{Shaded}
\begin{Highlighting}[]
\CommentTok{\# Observe the saved plots to analyze behavior}
\FunctionTok{cat}\NormalTok{(}\StringTok{"All forecast plots have been saved as PNG files in the working directory.}\SpecialCharTok{\textbackslash{}n}\StringTok{"}\NormalTok{)}
\end{Highlighting}
\end{Shaded}

\begin{verbatim}
## All forecast plots have been saved as PNG files in the working directory.
\end{verbatim}

The ARIMA model provides a strong forecast of Australian air passengers
based on historical data from 1970 to 2011. The model aligns well with
the data's upward trend and projects continued growth. The confidence
intervals indicate increasing uncertainty further into the future, but
overall, the model suggests robust forecasting.

The ARIMA(0,1,0) model with drift, which forecasts a steady linear
increase in passenger numbers. While this model captures the general
growth trend, it lacks the flexibility of the selected ARIMA model, as
it does not account for more complex dynamics present in the historical
data.

The ARIMA(2,1,2) model with drift. This model offers a more nuanced
representation of the passenger data, capturing both growth and some
additional fluctuations. When the drift is removed, as shown in the
fourth plot (ARIMA(2,1,2) without drift), the forecasts become stagnant,
indicating the importance of the drift term for projecting future
increases in passenger numbers.

The fifth plot illustrates an ARIMA(0,2,1) model with a constant. This
model produces forecasts with a steeper growth trajectory compared to
other models, emphasizing how the second-order differencing combined
with a constant can amplify the growth trend.

\section{9.8 )}\label{section-6}

For the United States GDP series (from global\_economy):

\begin{enumerate}
\def\labelenumi{\alph{enumi})}
\tightlist
\item
  if necessary, find a suitable Box-Cox transformation for the data;
\item
  fit a suitable ARIMA model to the transformed data using ARIMA();
\item
  try some other plausible models by experimenting with the orders
  chosen;
\item
  choose what you think is the best model and check the residual
  diagnostics;
\item
  produce forecasts of your fitted model. Do the forecasts look
  reasonable?
\item
  compare the results with what you would obtain using ETS() (with no
  transformation).
\end{enumerate}

\begin{Shaded}
\begin{Highlighting}[]
\CommentTok{\# Load necessary libraries}
\FunctionTok{library}\NormalTok{(fpp3)}

\CommentTok{\# Filter United States GDP data}
\NormalTok{us\_gdp }\OtherTok{\textless{}{-}}\NormalTok{ global\_economy }\SpecialCharTok{|\textgreater{}} \FunctionTok{filter}\NormalTok{(Country }\SpecialCharTok{==} \StringTok{"United States"}\NormalTok{)}

\CommentTok{\# Part (a): Box{-}Cox transformation if necessary}
\NormalTok{lambda }\OtherTok{\textless{}{-}}\NormalTok{ us\_gdp }\SpecialCharTok{|\textgreater{}} 
  \FunctionTok{features}\NormalTok{(GDP, }\AttributeTok{features =}\NormalTok{ guerrero) }\SpecialCharTok{|\textgreater{}} 
  \FunctionTok{pull}\NormalTok{(lambda\_guerrero)}
\FunctionTok{cat}\NormalTok{(}\StringTok{"The optimal Box{-}Cox lambda value is:"}\NormalTok{, lambda, }\StringTok{"}\SpecialCharTok{\textbackslash{}n}\StringTok{"}\NormalTok{)}
\end{Highlighting}
\end{Shaded}

\begin{verbatim}
## The optimal Box-Cox lambda value is: 0.2819443
\end{verbatim}

\begin{Shaded}
\begin{Highlighting}[]
\ControlFlowTok{if}\NormalTok{ (lambda }\SpecialCharTok{!=} \DecValTok{1}\NormalTok{) \{}
\NormalTok{  us\_gdp }\OtherTok{\textless{}{-}}\NormalTok{ us\_gdp }\SpecialCharTok{|\textgreater{}} \FunctionTok{mutate}\NormalTok{(}\AttributeTok{GDP =} \FunctionTok{box\_cox}\NormalTok{(GDP, lambda))}
\NormalTok{\}}

\CommentTok{\# Part (b): Fit a suitable ARIMA model}
\NormalTok{model\_arima }\OtherTok{\textless{}{-}}\NormalTok{ us\_gdp }\SpecialCharTok{|\textgreater{}} \FunctionTok{model}\NormalTok{(}\FunctionTok{ARIMA}\NormalTok{(GDP))}
\FunctionTok{report}\NormalTok{(model\_arima)}
\end{Highlighting}
\end{Shaded}

\begin{verbatim}
## Series: GDP 
## Model: ARIMA(1,1,0) w/ drift 
## 
## Coefficients:
##          ar1  constant
##       0.4586  118.1822
## s.e.  0.1198    9.5047
## 
## sigma^2 estimated as 5479:  log likelihood=-325.32
## AIC=656.65   AICc=657.1   BIC=662.78
\end{verbatim}

\begin{Shaded}
\begin{Highlighting}[]
\CommentTok{\# Part (c): Experiment with alternative ARIMA models}
\NormalTok{model\_arima\_alt1 }\OtherTok{\textless{}{-}}\NormalTok{ us\_gdp }\SpecialCharTok{|\textgreater{}} \FunctionTok{model}\NormalTok{(}\FunctionTok{ARIMA}\NormalTok{(GDP }\SpecialCharTok{\textasciitilde{}} \FunctionTok{pdq}\NormalTok{(}\DecValTok{1}\NormalTok{,}\DecValTok{1}\NormalTok{,}\DecValTok{0}\NormalTok{)))}
\NormalTok{model\_arima\_alt2 }\OtherTok{\textless{}{-}}\NormalTok{ us\_gdp }\SpecialCharTok{|\textgreater{}} \FunctionTok{model}\NormalTok{(}\FunctionTok{ARIMA}\NormalTok{(GDP }\SpecialCharTok{\textasciitilde{}} \FunctionTok{pdq}\NormalTok{(}\DecValTok{2}\NormalTok{,}\DecValTok{1}\NormalTok{,}\DecValTok{2}\NormalTok{)))}
\FunctionTok{report}\NormalTok{(model\_arima\_alt1)}
\end{Highlighting}
\end{Shaded}

\begin{verbatim}
## Series: GDP 
## Model: ARIMA(1,1,0) w/ drift 
## 
## Coefficients:
##          ar1  constant
##       0.4586  118.1822
## s.e.  0.1198    9.5047
## 
## sigma^2 estimated as 5479:  log likelihood=-325.32
## AIC=656.65   AICc=657.1   BIC=662.78
\end{verbatim}

\begin{Shaded}
\begin{Highlighting}[]
\FunctionTok{report}\NormalTok{(model\_arima\_alt2)}
\end{Highlighting}
\end{Shaded}

\begin{verbatim}
## Series: GDP 
## Model: ARIMA(2,1,2) w/ drift 
## 
## Coefficients:
##          ar1      ar2      ma1      ma2  constant
##       0.9557  -0.1011  -0.5094  -0.1386   31.2214
## s.e.  0.6096   0.4685   0.6043   0.3050    3.3800
## 
## sigma^2 estimated as 5734:  log likelihood=-325.05
## AIC=662.09   AICc=663.77   BIC=674.35
\end{verbatim}

\begin{Shaded}
\begin{Highlighting}[]
\CommentTok{\# Part (d): Manual Residual Diagnostics}

\CommentTok{\# Extract residuals as a tibble and convert to a numeric vector}
\NormalTok{residuals\_tibble }\OtherTok{\textless{}{-}}\NormalTok{ model\_arima }\SpecialCharTok{|\textgreater{}} \FunctionTok{residuals}\NormalTok{()}
\NormalTok{residuals\_vector }\OtherTok{\textless{}{-}}\NormalTok{ residuals\_tibble}\SpecialCharTok{$}\NormalTok{.resid}

\CommentTok{\# Plot residuals}
\NormalTok{p\_residuals }\OtherTok{\textless{}{-}} \FunctionTok{autoplot}\NormalTok{(}\FunctionTok{ts}\NormalTok{(residuals\_vector)) }\SpecialCharTok{+}
  \FunctionTok{labs}\NormalTok{(}\AttributeTok{title =} \StringTok{"Residuals of the Selected ARIMA Model"}\NormalTok{, }\AttributeTok{y =} \StringTok{"Residuals"}\NormalTok{) }\SpecialCharTok{+}
  \FunctionTok{theme\_minimal}\NormalTok{()}
\FunctionTok{print}\NormalTok{(p\_residuals)}
\end{Highlighting}
\end{Shaded}

\includegraphics{Assignment-6D624_files/figure-latex/98-1.pdf}

\begin{Shaded}
\begin{Highlighting}[]
\FunctionTok{ggsave}\NormalTok{(}\StringTok{"residuals\_manual\_plot.png"}\NormalTok{, }\AttributeTok{plot =}\NormalTok{ p\_residuals)}
\end{Highlighting}
\end{Shaded}

\begin{verbatim}
## Saving 6.5 x 4.5 in image
\end{verbatim}

\begin{Shaded}
\begin{Highlighting}[]
\CommentTok{\# Plot ACF of residuals}
\NormalTok{p\_acf }\OtherTok{\textless{}{-}} \FunctionTok{ggAcf}\NormalTok{(}\FunctionTok{ts}\NormalTok{(residuals\_vector)) }\SpecialCharTok{+}
  \FunctionTok{labs}\NormalTok{(}\AttributeTok{title =} \StringTok{"ACF of Residuals"}\NormalTok{) }\SpecialCharTok{+}
  \FunctionTok{theme\_minimal}\NormalTok{()}
\FunctionTok{print}\NormalTok{(p\_acf)}
\end{Highlighting}
\end{Shaded}

\includegraphics{Assignment-6D624_files/figure-latex/98-2.pdf}

\begin{Shaded}
\begin{Highlighting}[]
\FunctionTok{ggsave}\NormalTok{(}\StringTok{"acf\_manual\_residuals\_plot.png"}\NormalTok{, }\AttributeTok{plot =}\NormalTok{ p\_acf)}
\end{Highlighting}
\end{Shaded}

\begin{verbatim}
## Saving 6.5 x 4.5 in image
\end{verbatim}

\begin{Shaded}
\begin{Highlighting}[]
\CommentTok{\# Perform the Ljung{-}Box test}
\NormalTok{ljung\_box }\OtherTok{\textless{}{-}} \FunctionTok{Box.test}\NormalTok{(residuals\_vector, }\AttributeTok{lag =} \DecValTok{10}\NormalTok{, }\AttributeTok{type =} \StringTok{"Ljung{-}Box"}\NormalTok{)}
\FunctionTok{print}\NormalTok{(ljung\_box)}
\end{Highlighting}
\end{Shaded}

\begin{verbatim}
## 
##  Box-Ljung test
## 
## data:  residuals_vector
## X-squared = 3.8137, df = 10, p-value = 0.9554
\end{verbatim}

\begin{Shaded}
\begin{Highlighting}[]
\CommentTok{\# Assess results of the Ljung{-}Box test}
\ControlFlowTok{if}\NormalTok{ (ljung\_box}\SpecialCharTok{$}\NormalTok{p.value }\SpecialCharTok{\textgreater{}} \FloatTok{0.05}\NormalTok{) \{}
  \FunctionTok{cat}\NormalTok{(}\StringTok{"Residuals resemble white noise (no significant autocorrelation).}\SpecialCharTok{\textbackslash{}n}\StringTok{"}\NormalTok{)}
\NormalTok{\} }\ControlFlowTok{else}\NormalTok{ \{}
  \FunctionTok{cat}\NormalTok{(}\StringTok{"Residuals show significant autocorrelation. Consider revising the model.}\SpecialCharTok{\textbackslash{}n}\StringTok{"}\NormalTok{)}
\NormalTok{\}}
\end{Highlighting}
\end{Shaded}

\begin{verbatim}
## Residuals resemble white noise (no significant autocorrelation).
\end{verbatim}

\begin{Shaded}
\begin{Highlighting}[]
\CommentTok{\# Part (e): Produce forecasts of the fitted model}
\NormalTok{forecasts }\OtherTok{\textless{}{-}}\NormalTok{ model\_arima }\SpecialCharTok{|\textgreater{}} \FunctionTok{forecast}\NormalTok{(}\AttributeTok{h =} \DecValTok{10}\NormalTok{)}
\NormalTok{p\_forecast }\OtherTok{\textless{}{-}} \FunctionTok{autoplot}\NormalTok{(forecasts, us\_gdp) }\SpecialCharTok{+}
  \FunctionTok{labs}\NormalTok{(}\AttributeTok{title =} \StringTok{"ARIMA Model Forecasts"}\NormalTok{, }\AttributeTok{y =} \StringTok{"GDP"}\NormalTok{) }\SpecialCharTok{+}
  \FunctionTok{theme\_minimal}\NormalTok{()}
\FunctionTok{print}\NormalTok{(p\_forecast)}
\end{Highlighting}
\end{Shaded}

\includegraphics{Assignment-6D624_files/figure-latex/98-3.pdf}

\begin{Shaded}
\begin{Highlighting}[]
\FunctionTok{ggsave}\NormalTok{(}\StringTok{"forecast\_arima.png"}\NormalTok{, }\AttributeTok{plot =}\NormalTok{ p\_forecast)}
\end{Highlighting}
\end{Shaded}

\begin{verbatim}
## Saving 6.5 x 4.5 in image
\end{verbatim}

\begin{Shaded}
\begin{Highlighting}[]
\CommentTok{\# Part (f): Compare with forecasts obtained using ETS() (with no transformation)}
\NormalTok{model\_ets }\OtherTok{\textless{}{-}}\NormalTok{ us\_gdp }\SpecialCharTok{|\textgreater{}} \FunctionTok{model}\NormalTok{(}\FunctionTok{ETS}\NormalTok{(GDP))}
\NormalTok{forecasts\_ets }\OtherTok{\textless{}{-}}\NormalTok{ model\_ets }\SpecialCharTok{|\textgreater{}} \FunctionTok{forecast}\NormalTok{(}\AttributeTok{h =} \DecValTok{10}\NormalTok{)}
\NormalTok{p\_forecast\_ets }\OtherTok{\textless{}{-}} \FunctionTok{autoplot}\NormalTok{(forecasts\_ets, us\_gdp) }\SpecialCharTok{+}
  \FunctionTok{labs}\NormalTok{(}\AttributeTok{title =} \StringTok{"ETS Model Forecasts"}\NormalTok{, }\AttributeTok{y =} \StringTok{"GDP"}\NormalTok{) }\SpecialCharTok{+}
  \FunctionTok{theme\_minimal}\NormalTok{()}
\FunctionTok{print}\NormalTok{(p\_forecast\_ets)}
\end{Highlighting}
\end{Shaded}

\includegraphics{Assignment-6D624_files/figure-latex/98-4.pdf}

\begin{Shaded}
\begin{Highlighting}[]
\FunctionTok{ggsave}\NormalTok{(}\StringTok{"forecast\_ets.png"}\NormalTok{, }\AttributeTok{plot =}\NormalTok{ p\_forecast\_ets)}
\end{Highlighting}
\end{Shaded}

\begin{verbatim}
## Saving 6.5 x 4.5 in image
\end{verbatim}

\begin{Shaded}
\begin{Highlighting}[]
\CommentTok{\# Comparison of ARIMA and ETS forecasts}
\NormalTok{forecasts\_combined }\OtherTok{\textless{}{-}} \FunctionTok{bind\_rows}\NormalTok{(}
\NormalTok{  forecasts }\SpecialCharTok{|\textgreater{}} \FunctionTok{mutate}\NormalTok{(}\AttributeTok{Model =} \StringTok{"ARIMA"}\NormalTok{),}
\NormalTok{  forecasts\_ets }\SpecialCharTok{|\textgreater{}} \FunctionTok{mutate}\NormalTok{(}\AttributeTok{Model =} \StringTok{"ETS"}\NormalTok{)}
\NormalTok{)}
\NormalTok{p\_forecast\_combined }\OtherTok{\textless{}{-}} \FunctionTok{autoplot}\NormalTok{(forecasts\_combined, us\_gdp) }\SpecialCharTok{+}
  \FunctionTok{labs}\NormalTok{(}\AttributeTok{title =} \StringTok{"Comparison of ARIMA and ETS Forecasts"}\NormalTok{, }\AttributeTok{y =} \StringTok{"GDP"}\NormalTok{) }\SpecialCharTok{+}
  \FunctionTok{facet\_wrap}\NormalTok{(}\SpecialCharTok{\textasciitilde{}}\NormalTok{Model, }\AttributeTok{scales =} \StringTok{"free\_y"}\NormalTok{) }\SpecialCharTok{+}
  \FunctionTok{theme\_minimal}\NormalTok{()}
\FunctionTok{print}\NormalTok{(p\_forecast\_combined)}
\end{Highlighting}
\end{Shaded}

\includegraphics{Assignment-6D624_files/figure-latex/98-5.pdf}

\begin{Shaded}
\begin{Highlighting}[]
\FunctionTok{ggsave}\NormalTok{(}\StringTok{"forecast\_combined.png"}\NormalTok{, }\AttributeTok{plot =}\NormalTok{ p\_forecast\_combined)}
\end{Highlighting}
\end{Shaded}

\begin{verbatim}
## Saving 6.5 x 4.5 in image
\end{verbatim}

\section{Interpretation}\label{interpretation}

The Residuals are the differences between the observed and predicted
values, and this plot helps evaluate whether these errors are randomly
distributed around zero. In an ideal scenario, residuals should exhibit
no discernible patterns, as this indicates the model has captured all
the significant dynamics of the data. If clusters or trends are visible,
it suggests that the model is missing key components. This plot
generally aligns with the expectation of randomness, validating the
ARIMA model's effectiveness in fitting the U.S. GDP data.

ACF of Residuals:

The autocorrelation function (ACF) of residuals, which measures the
correlation of the residuals at varying time lags where each bar
represents a lag, and those within the 95\% confidence bounds suggest no
significant autocorrelation. This indicates that residuals resemble
white noise. If bars exceed the bounds at certain lags, it signifies
autocorrelation, meaning the model may require refinements. The plot
supports the conclusion that the ARIMA model adequately accounts for the
GDP data's temporal dependencies, as most bars remain within the
confidence intervals.

ARIMA Model Forecasts: The ARIMA model is extending predictions 10
periods into the future, also the model projects a steady continuation
of the historical growth observed in the GDP series, with forecast
values closely aligning with past trends. Confidence intervals,
represented by shaded regions, widen as the forecast horizon extends,
signifying greater uncertainty in long-term predictions. This plot
highlights the ARIMA model's ability to capture and extrapolate the
underlying growth trajectory effectively while accounting for forecast
variability.

ETS Model Forecasts : This plot depicts forecasts produced by the ETS
model, which is specifically designed for data with exponential trends
and seasonality. The ETS model identifies the consistent growth pattern
in the GDP data and extends it into future periods, maintaining an
upward trajectory. Like the ARIMA model, the forecast includes
confidence intervals that expand as predictions go further into the
future. The results align well with historical data, showcasing the ETS
model's strength in capturing exponential trends and making plausible
predictions for the U.S. GDP series.

Comparison of ARIMA and ETS Forecasts : The comparison of forecasts from
the ARIMA and ETS model are both models exhibit similar upward growth
trends, but subtle differences in their forecast intervals and
trajectories are evident. For instance, the ARIMA model may produce
narrower confidence intervals, suggesting higher precision in
predictions. The ETS model, on the other hand, emphasizes exponential
characteristics in its forecasts. This comparison highlights the
strengths of each approach: ARIMA's flexibility in modeling temporal
dependencies versus ETS's focus on exponential trends. Together, they
provide valuable insights into future GDP growth, reinforcing the
reliability of both models for time series forecasting.

Note that the \texttt{echo\ =\ FALSE} parameter was added to the code
chunk to prevent printing of the R code that generated the plot.

\end{document}
